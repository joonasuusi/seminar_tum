% Tämä tässä on LaTeX-lähdekoodia. Prosenttimerkki aloittaa kommentin,
% jonka jälkeinen osa rivistä ei näy julkaistavassa dokumentissa.

% Kommentteja voi laittaa tarpeen mukaan - aivan kuin ohjelmakoodiin.
% Kaikki, mikä ei ole kommenttia, on LaTeX-järjestelmän tulkittavaksi
% tarkoitettua merkintää rakenteiselle dokumentille.

% LaTeX on kummallinen, mutta valtavan käyttökelpoinen vekotin, missä
% voit kirjoittaa ja siinä seassa tarpeen mukaan vähän ohjelmoida
% tuottaaksesi ladotusta tekstistä ja asemoiduista kuvista koostuvan
% julkaisun - ensialkuun nyt vaikkapa gradun. Kaikki on kovin
% automaattista ja näppärää varsinkin sitten, kun tekstin,
% lähdeviitteiden ja dokumentin sisäisten ristiviittausten määrä alkaa
% lähennellä mielekkään tieteellisen tutkimuksen tarpeita.

% Tämä pohja on muokattu gradu3 -dokumenttiluokan esimerkkigradusta
% ITKY4000:n tarpeisiin syksyllä 2019. Tästä on poistettu kaikki
% ylimääräinen, ja tähän on jätetty vain minimaalinen pohja ITKY4000:n
% oppimispäiväkirjan 4. osan toteuttamista varten.

% Tarkoitus on tarjota kaikille syksyn 2019 itkyttelijöille pehmeä
% ensikosketus LaTeXiin, jolla meidän omat kandimme ovatkin jo tehneet
% yhden tutkielman aiemmin.

% Ensikosketuksessa voit etsiä tästä tiedostosta kohdat, jotka pitää
% muuttaa, että mallina annetusta oppimispäiväkirjasta tulee esimerkin
% sijaan omasi. Sama prosessi tehdään yleensä teknisten ja
% matemaattisten alojen tutkimusartikkeleille: Ladataan lehden
% julkaisijan nettisivulta mallipohja ja aletaan muokata sitä
% kommenttien ohjaamana. Tulee just sen muotoinen artikkeli kuin
% kyseiseen lehteen halutaan. Ja tästä nyt tulee just sen muotoinen
% kuin ITKY4000:ssa halutaan :).

% Kun aikanaan aloitat oikean gradun tekemisen, hae netistä sen hetken
% uusin versio dokumenttiluokasta, laita pohjadokumenttisi
% mahdollisimman nopeasti versionhallintaan ja tee sille myös
% etävarasto (esimerkiksi privaatti repo githubiin, gitlabiin tmv.)
% ettei laitevika tai muu ulkoinen syy pääse tuhoamaan gradua varten
% tekemääsi työtä eivätkä versiot mene sekaisin. (Siinä vaiheessa
% koulutusta toivottavasti ymmärrät, mitä äskeinen sanahelinä
% tarkoittaa, ja osaat tehdä sen heittämällä sinä kauniina iltana, kun
% gradun aloitus tuntuu erityisen ajankohtaiselta:))

\documentclass[utf8,english]{gradu3}
% For any *foreign* student in there.. You may write in
% English, in which case you should use the following line instead:
%\documentclass[utf8,english]{gradu3}

% Suomalaiset kirjoittavat suomeksi! (==Finns write in Finnish!)

\usepackage{graphicx} % kuvien mukaan ottamista varten; tarvitaan.

%\usepackage{amsmath} % hyödyllinen jos tekstisi sisältää matikkaa,
%                     % ei pakollinen

% ... ja niin edelleen, oikeassa gradussa... tässä otetaan mukaan
% tarvittavia lisäosia erilaisten juttujen latomiseen dokumenttiin.

% HUOM! Tämän tulee olla viimeinen \usepackage koko dokumentissa!
\usepackage[bookmarksopen,bookmarksnumbered,linktocpage]{hyperref}

\addbibresource{ref.bib} % Lähdetietokannan tiedostonimi; tarvitaan.

% ^
% |
% +---- yllä olevat ovat teknisiä määritelmiä, joilla voi säätää kaikenlaista.

% LaTeX-dokumentin varsinainen sisältö alkaa aina \begin{document} ja
% päättyy \end{document}.

% Pro-tip: mallipohjan voi jättää LaTeX-lähdekoodin loppuun malliksi
% erilaisten juttujen tekemisestä teknisesti. Sieltä voi ammentaa
% copy-paste-modifyllä niin ei tarvitse niin paljon tehdä
% KVG-menettelyä LaTeX-temppujen tekemiseksi. Miten? Laittamalla
% vaan \end{document} jo siihen kohtaan, missä oma tämänhetkinen
% tuotos loppuu. LaTeX-järjestelmä lopettaa lukemisen käskystä siihen.

\begin{document}

% Seuraavaksi tulevat nimiötiedot täytyy aina laittaa. LaTeX-pohjat
% helpottavat muistamaan, mitkä nimiötiedot julkaisuun, esimerkiksi
% graduun, tarvitaan julkaisijan asettamien sääntöjen mukaan. Yleensä
% aloitetaan kirjoittaminen vaihtamalla nimiötiedot omiksi, ja
% ryhdytään sitten raapustelemaan sisältöä mallitekstinä olevan
% diibadaaban ja/tai muoto-ohjeistuksen tilalle.

\title{Security and privacy issues with call applications and browsers}
%\translatedtitle{}%

\author{Shashwat Mishra and Joonas Uusi-Autti} %  <-- oma nimesi tietysti tähän, ja yhteystiedot:
\contactinformation{\texttt{ge58tid@mytum.de, ge58wun@mytum.de}}
%\studyline{} % <-- voi vaihtaa oman tuohon.
\supervisor{Tibor Posa}
% --------------------------------------------------------------------
% Tiivistelmä ja avainsanat voivat jäädä ITKY4000:ssa samoiksi kuin
% esimerkissä, tai sitten voit halutessasi nostaa niihin esiin
% joitakin juuri omaan elämääsi liittyviä avainpohdintoja. Tiivistelmä
% on yleensä hyvin lyhyt ja kertoo vain kaikkein olennaisimman.


\abstract{Small research in the field of security and privacy concerning video conference applications
}

% Avainsanat on oltava yhdellä rivillä, josta voisi tulla tekstieditorissa pitkä:

\keywords{literature review, % <-- Pro tip: LaTeX ei näe rivinvaihtoa, kun on %
          security, %
          privacy, survey,% ... joten LaTeXille koko pätkä on yksi rivi
          video conference applications%
} % <--- Loppusulku!

\type{Course: Security and privacy economics} % oikeassa gradussa ei ole tätä riviä.
\subject{} % oikeassa gradussa tähän tulee oppiaine (oletuksena ''Tietotekniikan'')

% Nimiötiedot päättyvät tähän. Ne eivät tulostu mihinkään ennen komentoa \maketitle
% ----------------------------------------------------------------------------------

% Kansisivu tulostuu seuraavan komennon kohdalle aivan kuin gradussasi aikanaan:

\maketitle

% Seuraava komento ilmoittaa LaTeXille, että nyt vasta alkaa varsinainen sisältö.

\mainmatter


% Seuraavaa "sloppy paragraph"-komentoa ei sitten jätetä graduun tai
% artikkeliin! Se antaa LaTeXille luvan tehdä rumaa
% jälkeä.. oppimispäiväkirjassa ja muistiotyyppisissä dokumenteissa
% oikaistaan tässä kohtaa, kun jätetään käyttämättä aikaa viimeiseen
% viimeistelyvaiheeseen.. Kysy sitten aikanaan graduohjaajalta lisää
% siitä, mitä asiaan liittyy. Tai KVG...
\sloppypar 


% Gradun päätason otsikot laitetaan komennolla \chapter{otsikko}. Suomeksi
% näitä sanotaan luvuiksi.

\chapter{Introduction}
\label{introduction}
After the COVID-19 pandemic hit hard all over the world, video conference applications (VCA) usage has been increasing a lot in education and work environments and every user is not familiar with applications and their features \parencite{zsolt2020}. Especially education has been affected a lot. Remote-working has been a possibility, although not very common one. Remote-learning has not been common and due to COVID-19 pandemic every student and teacher had to familiarize new techniques. And by rushing into using these applications, it is possible that very rare group of students know what else VCA's do, besides the obvious, giving an opportunity to attend classes and so on. Data collecting is happening nowadays everywhere. It is clear that e.g. Zoom collects data \parencite{zoomdata}, but do students know what kind of data? And are students even concerned about data collecting?
%

We conducted a literature review [\ref{literatureReview}] and a questionnaire survey focusing on privacy issues with VCA's [\ref{vca}]. 
%

%
In the results [\ref{results}] section we open up our findings and (to be continued)

%
In the endof the research, reader can find discussion about the results and suggestions towards the future [\ref{discussion}].

% Ristiviittaukset tehdään komennolla
% \ref{kohteenKuvaavaNimiCaseSensitiveKirjaimilla} seuraavan esimerkin
% mukaisesti. Kivaa, selkeää, automaattista, ja lähellä koodarin
% normaalia ajatusmallia - eikö?

\chapter{Research aim and question}
\label{researchquestion}
Research aim is to collect and analyze data are students concerned about privacy issues while using video conference applications. Data collecting is done by questionnaire survey.

Research question is Security and privacy issues with call applications and browsers?


\chapter{Literature review}
\label{literatureReview}
Literature review is a survey of chosen area of study. It synthesizes the infor-
mation of selected area of literature, critically analyzing the information. Indications of gaps in knowledge, limitations in theories and new points of view may occur in process, which is done in scientific, organized way \parencite{literatureRoyal}.

\section{Privacy}
There has been a great number of articles in the past years concerning about the privacy issues with VCA's \parencite{vcadata}. It is clear for every user that these applications are collecting data from the users but the level of understanding what kind of data is collected and for what data is used for, is not as high as it probably should be.


\section{Security}

\chapter{Survey}
\label{vca}


\chapter{Results}
\label{results}



\chapter{Discussion}
\label{discussion}

\section{Suggestions}


%
% <--- Pro tip: LaTeXin kommenteilla voi tehdä itselleen tyhjän
%      näköistä tilaa, joka ei tulostu dokumenttiin. Käytän tätä itse
%      aina tarvittaessa ajattelun apuvälineenä sekä yksittäisiin lauseisiin
%      kohdistuvina "FIXME: perustele väite tarkemmin tai poista löpinä
%      kokonaan" -tyylisinä kommentteina.
%
%      Auttaa kirjoitusprosessin vaihetta, jossa on (yleensä) tullut aika
%      "hyväksyä keskeneräisyys" ja siivota hieno TODO-lista backlogiin 
%      ja maton alle lopputuotteesta, jonka deadline lähestyy.
%


% Seuraavaksi merkataan alkamaan uusi luku taas
% komennolla \chapter{Kuvaava otsikko} ja sille annetaan
% ristiviittauksia varten nimi komennolla
% \label{kohteenKuvaavaNimiCaseSensitiveKirjaimilla}.



% Perusteksti kirjoitetaan samalla periaatteella kuin esimerkiksi
% MarkDownissa, eli mieluusti alle 80 merkin mittaisina riveinä.
% Tyhjä rivi merkitsee tekstikappaleen vaihtoa. Kappaleiden muotoilu
% ja tavutukset tulevat automaattisesti ja yleensä aika nätisti.

% Tarkoitus on, että tässä kohdassa poistat minun raapustuksen ja sen
% tilalle kopioit ja muotoilet nätisti oppimispäiväkirjan ensimmäisen
% osan tekstin. Sehän on jo tehty, mutta halutessasi voit myös muovata
% tekstiä tämän päivän ajattelusi mukaiseksi.

% Teknisesti muutostyö mennee jotakuinkin näin:
%
% 1. tekstiosuuden copy-paste toimisto-ohjelmasta
% 2. ylimääräiset rivinvaihdot kappaleiden väliin
% 3. kappaleiden rivitys tekstieditorin automaattisella rivityksellä
%    (esimerkiksi Notepad++:ssa se aiemmassa opetusvideossa etsitty
%    pikanäppäin)



% Korvaa otsikko omallasi sopivasti (ellet haaveile vallankumouksesta
% itsekin):

% Tieteellisessä tekstissä on tavanomainen tyyli kirjoittaa myös
% tekstinä auki kaikki havainnepiirrokset ja kaaviot. Harjoitellaan
% nyt sitä. Korvaa siis alla oleva teksti pohdinnalla omasta
% oppimispäiväkirjan 2. osassa tehdystä kuvasta. Ristiviittaus ja
% kuvan tekninen liittäminen LaTeX-dokumenttiin on annettu malliksi,
% samoin kuin esimerkki tavasta, jolla kuvan sisältöä voi avata.


% 8< -----------------------------------

% .. eli tähän asti oli tekstiä, jonka korvaat selityksellä omasta
% kuvastasi. Itse laitetaan, kuten on tyypillistä, piakkoin siitä
% kertovan tekstin *jälkeen*. Tällä tavoin laitetaan, ja kuvan
% asemointi annetaan LaTeXin automatiikan huoleksi:


  %
  % Ylläoleva koodi toimii sellaisenaan, jos tallennat kuvatiedostosi
  % PDF-muodossa tiedostoon nimeltä kaaviokuva.pdf esimerkkinä olleen 
  % kuvan tilalle.
  %
  % Alla on normaali \label{tunniste} kohdassa, johon se kuuluu.
  % Muuta suomenkielinen otsikko vastaamaan oman kuvasi sisältöä.
  %





% Tämä teksti on jo kirjoitettu Oppimispäiväkirjan 3. osassa.
%
% LaTeXia varten se tarvitsee muutaman säädön:
%
%  - Alilukujen otsikot merkitään \section{Kuvaava otsikko}
%
%  - Jos on alialilukuja, ne merkitään \subsection{Kuvaava otsikko}
%
%  - Tehdään teknisesti kunnollinen viittaus luettuun graduun komennolla
%    \textcite{bibtexAvain} tai \parencite{bibtexAvain}.
%
%  - Prosenttimerkit ja muut LaTeX-kielen syntaksiin kuuluvat jutut pitää 
%    laittaa jollain tavoin, minkä LaTeX ymmärtää (tai poistaa - less is
%    more:)).
%
%  - Suomalaiset lainausmerkit " on hyvä korvata kahdella heittomerkillä ''
%    niin tulee nätti.
%
%  - Tehdään muita pieniä säätöjä Markdownin ja LaTeXin syntaksien välillä.
%    Perusmallin esimerkkejä löytyy tekstini seasta:


% ------------------------------------------------------------------
%
% Tähän asti tuli muokkaamalla ja täydentämällä oppimispäiväkirjan
% osia 1-3. Seuraavaksi on 4. osan loppupohdinnan paikka.
%
% Korvaa esimerkit aliotsikoiden alta oman pohdintasi mukaisiksi.
%
% ------------------------------------------------------------------


\printbibliography

\end{document}
