% Tämä tässä on LaTeX-lähdekoodia. Prosenttimerkki aloittaa kommentin,
% jonka jälkeinen osa rivistä ei näy julkaistavassa dokumentissa.

% Kommentteja voi laittaa tarpeen mukaan - aivan kuin ohjelmakoodiin.
% Kaikki, mikä ei ole kommenttia, on LaTeX-järjestelmän tulkittavaksi
% tarkoitettua merkintää rakenteiselle dokumentille.

% LaTeX on kummallinen, mutta valtavan käyttökelpoinen vekotin, missä
% voit kirjoittaa ja siinä seassa tarpeen mukaan vähän ohjelmoida
% tuottaaksesi ladotusta tekstistä ja asemoiduista kuvista koostuvan
% julkaisun - ensialkuun nyt vaikkapa gradun. Kaikki on kovin
% automaattista ja näppärää varsinkin sitten, kun tekstin,
% lähdeviitteiden ja dokumentin sisäisten ristiviittausten määrä alkaa
% lähennellä mielekkään tieteellisen tutkimuksen tarpeita.

% Tämä pohja on muokattu gradu3 -dokumenttiluokan esimerkkigradusta
% ITKY4000:n tarpeisiin syksyllä 2019. Tästä on poistettu kaikki
% ylimääräinen, ja tähän on jätetty vain minimaalinen pohja ITKY4000:n
% oppimispäiväkirjan 4. osan toteuttamista varten.

% Tarkoitus on tarjota kaikille syksyn 2019 itkyttelijöille pehmeä
% ensikosketus LaTeXiin, jolla meidän omat kandimme ovatkin jo tehneet
% yhden tutkielman aiemmin.

% Ensikosketuksessa voit etsiä tästä tiedostosta kohdat, jotka pitää
% muuttaa, että mallina annetusta oppimispäiväkirjasta tulee esimerkin
% sijaan omasi. Sama prosessi tehdään yleensä teknisten ja
% matemaattisten alojen tutkimusartikkeleille: Ladataan lehden
% julkaisijan nettisivulta mallipohja ja aletaan muokata sitä
% kommenttien ohjaamana. Tulee just sen muotoinen artikkeli kuin
% kyseiseen lehteen halutaan. Ja tästä nyt tulee just sen muotoinen
% kuin ITKY4000:ssa halutaan :).

% Kun aikanaan aloitat oikean gradun tekemisen, hae netistä sen hetken
% uusin versio dokumenttiluokasta, laita pohjadokumenttisi
% mahdollisimman nopeasti versionhallintaan ja tee sille myös
% etävarasto (esimerkiksi privaatti repo githubiin, gitlabiin tmv.)
% ettei laitevika tai muu ulkoinen syy pääse tuhoamaan gradua varten
% tekemääsi työtä eivätkä versiot mene sekaisin. (Siinä vaiheessa
% koulutusta toivottavasti ymmärrät, mitä äskeinen sanahelinä
% tarkoittaa, ja osaat tehdä sen heittämällä sinä kauniina iltana, kun
% gradun aloitus tuntuu erityisen ajankohtaiselta:))

\documentclass[utf8,english]{gradu3}
% For any *foreign* student in there.. You may write in
% English, in which case you should use the following line instead:
%\documentclass[utf8,english]{gradu3}

% Suomalaiset kirjoittavat suomeksi! (==Finns write in Finnish!)

\usepackage{graphicx} % kuvien mukaan ottamista varten; tarvitaan.

%\usepackage{amsmath} % hyödyllinen jos tekstisi sisältää matikkaa,
%                     % ei pakollinen

% ... ja niin edelleen, oikeassa gradussa... tässä otetaan mukaan
% tarvittavia lisäosia erilaisten juttujen latomiseen dokumenttiin.

% HUOM! Tämän tulee olla viimeinen \usepackage koko dokumentissa!
\usepackage[bookmarksopen,bookmarksnumbered,linktocpage]{hyperref}

\addbibresource{ref.bib} % Lähdetietokannan tiedostonimi; tarvitaan.

% ^
% |
% +---- yllä olevat ovat teknisiä määritelmiä, joilla voi säätää kaikenlaista.

% LaTeX-dokumentin varsinainen sisältö alkaa aina \begin{document} ja
% päättyy \end{document}.

% Pro-tip: mallipohjan voi jättää LaTeX-lähdekoodin loppuun malliksi
% erilaisten juttujen tekemisestä teknisesti. Sieltä voi ammentaa
% copy-paste-modifyllä niin ei tarvitse niin paljon tehdä
% KVG-menettelyä LaTeX-temppujen tekemiseksi. Miten? Laittamalla
% vaan \end{document} jo siihen kohtaan, missä oma tämänhetkinen
% tuotos loppuu. LaTeX-järjestelmä lopettaa lukemisen käskystä siihen.

\begin{document}

% Seuraavaksi tulevat nimiötiedot täytyy aina laittaa. LaTeX-pohjat
% helpottavat muistamaan, mitkä nimiötiedot julkaisuun, esimerkiksi
% graduun, tarvitaan julkaisijan asettamien sääntöjen mukaan. Yleensä
% aloitetaan kirjoittaminen vaihtamalla nimiötiedot omiksi, ja
% ryhdytään sitten raapustelemaan sisältöä mallitekstinä olevan
% diibadaaban ja/tai muoto-ohjeistuksen tilalle.

\title{Security and privacy issues with video conference applications}
%\translatedtitle{}%

\author{Shashwat Mishra and Joonas Uusi-Autti} %  <-- oma nimesi tietysti tähän, ja yhteystiedot:
\contactinformation{\texttt{ge58tid@mytum.de, ge58wun@mytum.de}}
%\studyline{} % <-- voi vaihtaa oman tuohon.
\supervisor{Tibor Posa}
% --------------------------------------------------------------------
% Tiivistelmä ja avainsanat voivat jäädä ITKY4000:ssa samoiksi kuin
% esimerkissä, tai sitten voit halutessasi nostaa niihin esiin
% joitakin juuri omaan elämääsi liittyviä avainpohdintoja. Tiivistelmä
% on yleensä hyvin lyhyt ja kertoo vain kaikkein olennaisimman.


\abstract{Small research in the field of security and privacy concerning video conference applications
}

% Avainsanat on oltava yhdellä rivillä, josta voisi tulla tekstieditorissa pitkä:

\keywords{literature review, % <-- Pro tip: LaTeX ei näe rivinvaihtoa, kun on %
          security, %
          privacy, survey,% ... joten LaTeXille koko pätkä on yksi rivi
          video conference applications,survey%
} % <--- Loppusulku!

\type{Course: Security and privacy economics} % oikeassa gradussa ei ole tätä riviä.
\subject{} % oikeassa gradussa tähän tulee oppiaine (oletuksena ''Tietotekniikan'')

% Nimiötiedot päättyvät tähän. Ne eivät tulostu mihinkään ennen komentoa \maketitle
% ----------------------------------------------------------------------------------

% Kansisivu tulostuu seuraavan komennon kohdalle aivan kuin gradussasi aikanaan:

\maketitle

% Seuraava komento ilmoittaa LaTeXille, että nyt vasta alkaa varsinainen sisältö.

\mainmatter


% Seuraavaa "sloppy paragraph"-komentoa ei sitten jätetä graduun tai
% artikkeliin! Se antaa LaTeXille luvan tehdä rumaa
% jälkeä.. oppimispäiväkirjassa ja muistiotyyppisissä dokumenteissa
% oikaistaan tässä kohtaa, kun jätetään käyttämättä aikaa viimeiseen
% viimeistelyvaiheeseen.. Kysy sitten aikanaan graduohjaajalta lisää
% siitä, mitä asiaan liittyy. Tai KVG...
\sloppypar 


% Gradun päätason otsikot laitetaan komennolla \chapter{otsikko}. Suomeksi
% näitä sanotaan luvuiksi.

\chapter{Introduction}
\label{introduction}
After the COVID-19 pandemic hit hard all over the world, video conference applications (VCA) usage has been increasing a lot in education and work environments and every user is not familiar with applications and their features \parencite{zsolt2020}. Especially education has been affected a lot. Remote-working has been a possibility, although not very common one. Remote-learning has not been common and due to COVID-19 pandemic every student and teacher had to familiarize new techniques. And by rushing into using these applications, it is possible that very rare group of students know what else VCA's do, besides the obvious, giving an opportunity to attend classes and learn. Data collecting is happening nowadays everywhere. It is clear that e.g. Zoom collects data \parencite{zoomdata}, but do students know what kind of data? And are students even concerned about data collecting?
Security problems happens everywhere, where there is network connection involved. High increase of using VCA's has also brought up some new security flaws to these applications. But what kind of issues and how organizations handle these new threats?
%

We conducted a literature review [\ref{literatureReview}] to gain more background information about the issues and a questionnaire survey focusing on privacy and security issues with VCA's [\ref{vca}]. Survey was directed to university students.
%

%
In the results [\ref{results}] section we open up our findings from the survey and observe how students use VCA's and how they react when facing some kind of security issues. And most importantly, are they aware of the data collecting which happens while using these kind of applications.

%
In the end of the research, reader can find discussion about the results and suggestions towards the future [\ref{discussion}]. These suggestions are solely based on our point of views, which are found in the research process and cannot be generalized, because of the fact that only three applications are being researched and answers came only from two universities.

% Ristiviittaukset tehdään komennolla
% \ref{kohteenKuvaavaNimiCaseSensitiveKirjaimilla} seuraavan esimerkin
% mukaisesti. Kivaa, selkeää, automaattista, ja lähellä koodarin
% normaalia ajatusmallia - eikö?

\chapter{Research aim and question}
\label{researchquestion}
Research aim is to collect and analyze data are students concerned about privacy and security issues while using video conference applications. Data collecting is done by questionnaire survey, using the surveymonkey platform.

Research question is Security and privacy issues with call applications?
Research is focusing on security and privacy problems only in the three biggest VCA applications Zoom, Google Meet and Microsoft Teams. These are are clearly most heavily used and out of the three, Zoom has the most users, 300 million and counting. While Microsoft Teams and Google Meet come clearly second and third, they are far behind from Zoom, when observing the user count.


\chapter{Literature review}
\label{literatureReview}
Literature review is a survey of chosen area of study. It synthesizes the infor-
mation of selected area of literature, critically analyzing the information. Indications of gaps in knowledge, limitations in theories and new points of view may occur in process, which is done in scientific, organized way \parencite{literatureRoyal}.

In this literature review part, research focuses on privacy point of view and security point of view.

Literature was gathered from Google Scholars and IEEE Xplorer research databases. With 12 different keyword combinations from these two databases, it was possible to gather enough articles for the literature review. It was clear that most researched application was Zoom, mainly because it has been growing to be the biggest platform during the COVID19 pandemic.

\section{Privacy of video conference applications}
There has been a great number of articles in the past years concerning about the privacy issues with VCA's \parencite{vcadata}. It is clear for every user that these applications are collecting data from the users but the level of understanding what kind of data is collected and for what data is used for, is not as high as it probably should be.

While the Zoom application is the most used VCA there is, other video conference applications are also vulnerable to privacy issues. Every VCA has different privacy policies but they don't differ that much and basically none of these options aren't great \parencite{ZoomGoogleMS}. And according to these three biggest companies (Zoom, Google, Microsoft) privacy policies, all of the mentioned companies can collect data when using these applications. 

Although every one of these companies stated that they respect customers privacy, has Zoom for example faced investigations towards their privacy practices and especially data privacy laws in the United States makes it hard to defend customers against Zoom's privacy practices \parencite{Darkside}. But like stated before it is not just Zoom, problem involves also other companies.

One of the biggest problems concerning privacy policies is that they are very rarely read. And that causes problems to user's data. For example in Zoom, its administrators can see detailed information on how, when and where users are. Also they have access to calls which have been recorded and they can basically join any call at any time \parencite{Darkside}.
%
While all this is happening to you, it is also happening to your contacts. Zoom can collect your contact lists, facebook profiles and contacts facebook profiles. This is all stated in the privacy policies but like stated before, very few people read them and even fewer actually understands the terms.

Now, during the COVID19 pandemic, when people have started to use more of these VCA's, malicious users have had time of their lives. Especially when some conference meeting links have been public on some website. These malicious users have been using machine learning techiques to infiltrate to the meetings and when in a meeting, collected personal data from users. Data such as profile pictures, usernames, voice and personal data which has been shared in the meetings \parencite{zoomingInto}.
Malicious users can also, with these informations, possibly predict users interests, activities and even social security numbers.

These privacy issues can lead to serious risks and damage the companies or inviduals. And how the issues concern users and are users usually aware of them, will be surveyed here \ref{vca}.

\section{Security of video conference applications}
Since the COVID19 pandemic has been tearing the earth, video conference applications usage has been increasing tremendiously. And usually when a application or software generates popularity to such an extend (e.g. Zoom went from 10 million daily users to 300 million daily users \parencite{goneWrongZoom}), there is usually security flaws involved. 
In march 2020, when basically everything went online (work,education), different organizations started to use different platforms. Like stated before, three biggest platform considering VCA's, are Google Meet, Zoom and Microsoft Teams.

One of most common security issues was end-to-end-encryption(E2EE), which allowed other users in the meeting see lot about you and some people could "Zoombomb", these meetings \parencite{goneWrongZoom}. "Zoombombing" is a phenomenon where malicious user "hijacks" the meeting and shares content which is disruptive for users. End-to-end-encryption is a scheme for communication for VCA's and other messaging applications in which only user in that precise meeting can only send and receive messages. For "Zoombombing", this is the most important security scheme \parencite{e2eeZoom}.
Out of these three VCA applications, only Zoom provides E2EE but only in text chat and when sharing files. Google Meet and Microsoft Teams do not provide E2EE in any of the following criterias: text chat, voice calls, video calls, file sharing and screen sharing \parencite{safetyOfVca}. While these findings are from 2020, E2EE functions may have been upgraded.

While these three biggest VCA platforms have similiraties, there are some differencies how these three applications operate. User can use every one of them via browser but Google Meet operates solely in browser and this feature separates Meet from the other two. Google has said that it limits the attack surface of their platform, because they can immediately deploy changes, since it is only accesible on the browser. In addition, Meet requires Single Sign-On(SSO) with their Google accounts, when joining the meeting. Usually Google accounts consists of two-step verification and with these functios, Google focuses on preventing phishing, account hacking and similar attacks as "Zoomboming" \parencite{dynamicSecurityAnalysis}.

\chapter{Survey}
\label{vca}
\section{Preparing the survey}
Survey's questions were conducted from the gathered literature and they were made easy to answer, to get enough data. Survey was sent to studentgroups to two different universities, University of Jyväskylä and Technical University of Munich. 

Answers were submitted anonymously, so the analyzed data is general. Although probably there would have been some differencies within the answers, considering the studyculture, which is a bit different in these two countries.

Questionnaire was online for two weeks for students to answer. N was XX and the questionnaire consisted of 10 questions, privacy and security issues with video conference applications. We will examine the questions more in detail in the next section.

\section{Question template}
In this section you can find questions and short explanation behind the question.

1. Are you aware of data collecting when using video conference applications?(Zoom, Microsoft Teams, Google Meet) Answers: Yes/No. I would assume that not that many students are not aware data collecting while using VCA's.

2. Are you, as a student, concerned about data collecting when using video conference applications?(Zoom, Microsoft Teams, Google Meet) Answers: Yes/No. Follow-up question for the fist one. 

3. How concerned you are about data collecting? (1 = Not at all, 4 = Very concerned) Four options to see are the students concerned.

4. What video conference application you use the most? Options are Zoom, Meets, Teams and Other, please specify. Here one sees what platforms students use the most. 

5. How often do you read applications privacy policy text? 5 options. It is documented that truthfully speaking, it is rare to read applications privacy policy.

6. What kind of data you think video conference applications collect from users? Open question. Gathering assumptions of what kind of data platforms collect.

Security part of the questionnaire.

7. Do you trust video conference applications security functions? Yes/No. Users trust towards applications.

8. When organizing a meeting, do you protect the meeting with password? Yes/No. Password protection is one of the first ways to protect one's meeting. During the pandemic, this has become more frequent way.

9. What kind of other security measurements you take into consideration when attending/organizing a meeting? (waiting room, link sharing, etc.) Open question. There are other ways of protecting a meeting than a password. Are students familiar with them?

10. What kind of security threats do you think are most common with video conference applications? Open question. Gather information about what students think are the most common security threats.

With these questions appointed to students, we can hopefully see a pattern on how students react to privacy and security issues. And something about their knowledge towards these issues. 

\chapter{Survey results}
\label{surveyresults}
Survey was analyzed question by question, and bigger themes were combined also to one bigger picture. Number of survey participants was XX, and due to a number so small, these results are more of a first scratch or guideline giving results. Participants were residents of Finland or Germany. Nationalities differ from the place of living. This survey can give insights of the question are students aware of privacy and security issues with video conference applications.

\section{Analyzing the answers}


\chapter{Discussion}
\label{discussion}

\section{Suggestions}


%
% <--- Pro tip: LaTeXin kommenteilla voi tehdä itselleen tyhjän
%      näköistä tilaa, joka ei tulostu dokumenttiin. Käytän tätä itse
%      aina tarvittaessa ajattelun apuvälineenä sekä yksittäisiin lauseisiin
%      kohdistuvina "FIXME: perustele väite tarkemmin tai poista löpinä
%      kokonaan" -tyylisinä kommentteina.
%
%      Auttaa kirjoitusprosessin vaihetta, jossa on (yleensä) tullut aika
%      "hyväksyä keskeneräisyys" ja siivota hieno TODO-lista backlogiin 
%      ja maton alle lopputuotteesta, jonka deadline lähestyy.
%


% Seuraavaksi merkataan alkamaan uusi luku taas
% komennolla \chapter{Kuvaava otsikko} ja sille annetaan
% ristiviittauksia varten nimi komennolla
% \label{kohteenKuvaavaNimiCaseSensitiveKirjaimilla}.



% Perusteksti kirjoitetaan samalla periaatteella kuin esimerkiksi
% MarkDownissa, eli mieluusti alle 80 merkin mittaisina riveinä.
% Tyhjä rivi merkitsee tekstikappaleen vaihtoa. Kappaleiden muotoilu
% ja tavutukset tulevat automaattisesti ja yleensä aika nätisti.

% Tarkoitus on, että tässä kohdassa poistat minun raapustuksen ja sen
% tilalle kopioit ja muotoilet nätisti oppimispäiväkirjan ensimmäisen
% osan tekstin. Sehän on jo tehty, mutta halutessasi voit myös muovata
% tekstiä tämän päivän ajattelusi mukaiseksi.

% Teknisesti muutostyö mennee jotakuinkin näin:
%
% 1. tekstiosuuden copy-paste toimisto-ohjelmasta
% 2. ylimääräiset rivinvaihdot kappaleiden väliin
% 3. kappaleiden rivitys tekstieditorin automaattisella rivityksellä
%    (esimerkiksi Notepad++:ssa se aiemmassa opetusvideossa etsitty
%    pikanäppäin)



% Korvaa otsikko omallasi sopivasti (ellet haaveile vallankumouksesta
% itsekin):

% Tieteellisessä tekstissä on tavanomainen tyyli kirjoittaa myös
% tekstinä auki kaikki havainnepiirrokset ja kaaviot. Harjoitellaan
% nyt sitä. Korvaa siis alla oleva teksti pohdinnalla omasta
% oppimispäiväkirjan 2. osassa tehdystä kuvasta. Ristiviittaus ja
% kuvan tekninen liittäminen LaTeX-dokumenttiin on annettu malliksi,
% samoin kuin esimerkki tavasta, jolla kuvan sisältöä voi avata.


% 8< -----------------------------------

% .. eli tähän asti oli tekstiä, jonka korvaat selityksellä omasta
% kuvastasi. Itse laitetaan, kuten on tyypillistä, piakkoin siitä
% kertovan tekstin *jälkeen*. Tällä tavoin laitetaan, ja kuvan
% asemointi annetaan LaTeXin automatiikan huoleksi:


  %
  % Ylläoleva koodi toimii sellaisenaan, jos tallennat kuvatiedostosi
  % PDF-muodossa tiedostoon nimeltä kaaviokuva.pdf esimerkkinä olleen 
  % kuvan tilalle.
  %
  % Alla on normaali \label{tunniste} kohdassa, johon se kuuluu.
  % Muuta suomenkielinen otsikko vastaamaan oman kuvasi sisältöä.
  %





% Tämä teksti on jo kirjoitettu Oppimispäiväkirjan 3. osassa.
%
% LaTeXia varten se tarvitsee muutaman säädön:
%
%  - Alilukujen otsikot merkitään \section{Kuvaava otsikko}
%
%  - Jos on alialilukuja, ne merkitään \subsection{Kuvaava otsikko}
%
%  - Tehdään teknisesti kunnollinen viittaus luettuun graduun komennolla
%    \textcite{bibtexAvain} tai \parencite{bibtexAvain}.
%
%  - Prosenttimerkit ja muut LaTeX-kielen syntaksiin kuuluvat jutut pitää 
%    laittaa jollain tavoin, minkä LaTeX ymmärtää (tai poistaa - less is
%    more:)).
%
%  - Suomalaiset lainausmerkit " on hyvä korvata kahdella heittomerkillä ''
%    niin tulee nätti.
%
%  - Tehdään muita pieniä säätöjä Markdownin ja LaTeXin syntaksien välillä.
%    Perusmallin esimerkkejä löytyy tekstini seasta:


% ------------------------------------------------------------------
%
% Tähän asti tuli muokkaamalla ja täydentämällä oppimispäiväkirjan
% osia 1-3. Seuraavaksi on 4. osan loppupohdinnan paikka.
%
% Korvaa esimerkit aliotsikoiden alta oman pohdintasi mukaisiksi.
%
% ------------------------------------------------------------------


\printbibliography

\end{document}
